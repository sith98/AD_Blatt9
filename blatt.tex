\documentclass[a4paper]{article}
\usepackage[utf8]{inputenc}
\usepackage[T1]{fontenc}
\usepackage[ngerman]{babel}
\usepackage{amsmath}
\usepackage{amssymb}
\usepackage{hyperref}

\title{AD -- Blatt 9}
\author{Simon Thelen}

\begin{document}
    \maketitle

    \section*{Aufgabe 1}
    \begin{equation*}
        \begin{array}{ll}
            s & \operatorname{h}(s) \\
            \hline
            61 & 700 \\
            62 & 318 \\
            63 & 936 \\
            64 & 554 \\
            65 & 172
        \end{array}
    \end{equation*}

    \section*{Aufgabe 2}
    $P_k$ ist die Wahrscheinlichkeit, dass $k$ Werte der Hashtable beim ersten Versuch in einen gegebenen Bucket eingefügt werden sollten.

    \section*{Aufgabe 4}
    \subsection*{Teilaufgabe 1}
    \label{subsec:aufgabe4_1}

    Sei $\operatorname{s}(n) = \sum_{k = 1}^{i}{k}$.

    \begin{enumerate}
        \item Berechne für alle $i \in [1, n]$ das Wert-pro-Volumen-Verhältnis $r_i = \frac{w_i}{v_i}$.
        (\textit{Laufzeit: $\operatorname{O}(n)$})

        \item Sortiere absteigend nach $r_i$.
        Gegenstände sind jetzt neu von $1$ bis $n$ durchnummeriert.
        (\textit{Laufzeit: $\operatorname{O}(n \log{n})$})

        \item Setze alle $a_i$ auf $0$.
        (\textit{Laufzeit: $\operatorname{O}(n)$})

        \item Gehe alle Gegenstände von $1$ bis $n$ durch.
        (\textit{Laufzeit: $\operatorname{O}(n)$})
        \begin{enumerate}
            \item Setze $a_i$ auf $1$, solange $W \leq \operatorname{s}(i)$
            \item Wenn $\operatorname{s}(i - 1) < W < \operatorname{s}(i)$, setze $s_i$ auf $\frac{v_i}{W - \operatorname{s}(i - 1)}$ und beende den Algorithmus.
        \end{enumerate}
    \end{enumerate}


    \section*{Teilaufgabe 2}
    \begin{gather*}
        w_1 = 5, w_2 = 3, w_3 = 3 \\
        v_1 = 4, v_2 = 3, v_3 = 3
    \end{gather*}

    Der in~\nameref{subsec:aufgabe4_1} beschriebene Algorithmus würde nur $w_1$ mit Wert $5$ verwenden, wenn jedes $a_i \notin \mathbb{Z}$ durch $0$ ersetzt wird.
    Die perfekte Lösung wäre jedoch in diesem Fall $w_2$ und $w_3$ mit Gesamtwert $6$.


\end{document}